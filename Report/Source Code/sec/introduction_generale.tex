\clearpage
\phantomsection
\addcontentsline{toc}{chapter}{INTRODUCTION GÉNÉRALE}
\chapter*{INTRODUCTION GÉNÉRALE}

\lettrine[lines=3,lraise=0.1,findent=0.6em]{L'}{analyse} des sentiments, également connue sous le nom de mining d’opinions, est le processus qui consiste à extraire des informations subjectives, à partir des textes, généralement des opinions, des émotions, ou des attitudes exprimées, par des individus à l’égard d’un sujet particulier. Avec la prolifération des réseaux sociaux, cette discipline est devenue essentielle pour comprendre les tendances et les opinions qui circulent en ligne. Aujourd’hui chaque page web dispose d’une section permettant aux utilisateurs de laisser leurs commentaires, ce qui n’était pas possible avant quelques années. Un exemple notable de ces pages web, Twitter ou bien X qu’est à l’origine une plateforme micro-blogging très populaire, Elle a connu une croissance régulière ces dernières années, est devenue un point de rencontre d’une gamme diversifiée des personnes (politiciens, professionnels, célébrités, étudiants, …), cette popularité se traduit par une énorme quantité d’informations couvrant une large gamme de sujets allant de bien-être vers les marques, la politique, les événements sociaux, ce qui rend Twitter un outil puissant pour extraire des tweets afin de faire des prédictions sur sentiments et les opinions des utilisateurs, ces applications présentent un intérêt commercial au entreprises et aux chercheurs pour prendre des décisions opérationnelle.


De nombreux modèles et services différents pour effectuer l'analyse des sentiments sont disponibles. Il est souvent difficile de choisir le bon pour le cas d'utilisation qui nous intéresse. Cette thèse analyse les techniques pertinentes qui ont été appliquées avec succès pour classifier la polarité des sentiments et propose une comparaison de leurs performances basée sur des expériences réalisées sur les ensembles de données Sentiment140-MV et Emotions. De plus, elle propose une analyse pour comprendre quand les modèles sont d'accord sur la classification correcte afin de mettre en évidence la marge d'amélioration possible en théorie. Trois grandes macro-catégories de modèles sont considérées : les modèles traditionnels basés sur des théorèmes mathématiques ou des intuitions (Naive Bayes, nu-SVC, Logistic Regression et AdaBoost), les modèles neuronaux (RNN, CNN et LSTM ). 

Afin de parvenir à ce but, cet étude est divisée en deux volets. Un volet théorique, ainsi d'un volet pratique. La partie théorique se compose de trois chapitres, le premier chapitre présente un cadrage générale du projet, le deuxième chapitre présente le fondament de l'analyse des sentiments et des émotions ainsi que les domaines ou ces pratiques sont appliqués. Nous éxplorons dans le troisème chapitre les méthodes et les approches utilisées pour l'analyse des sentiments et des émotions. La seconde partie, la partie pratique débute par une déscription des corpus, les étapes de l'analyse et une comparaison finale des résultats obtenus lors de l'évaluation. On finalise ce travail avec le developpement d'une application web basée sur le framework Django pour offrir une interface utilisateur intuitive et démontrer l'applicabilité pratique de cette analyse. 