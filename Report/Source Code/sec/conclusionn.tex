\clearpage
\phantomsection
\addcontentsline{toc}{chapter}{CONCLUSION ET PERSPECTIVES}
\chapter*{CONCLUSION ET PERSPECTIVES}
\lettrine[lines=1,lraise=0.1,findent=0.6em]{D}{ans} ce rapport, nous avons présenté notre projet d'analyse des sentiments et des émotions exprimés dans les tweets. Nos résultats ont montré que cette approche permet de mieux comprendre les réactions du public sur différents sujets. Cependant, des améliorations sont encore nécessaires, notamment dans la gestion de la négation, afin d'affiner davantage la compréhension des sentiments.

Une perspective intéressante serait de permettre à l'utilisateur de choisir le modèle d'analyse à appliquer depuis notre application, offrant ainsi une plus grande flexibilité. De plus, l'extension de notre analyse à d'autres sources de données, comme les commentaires de réseaux sociaux ou les avis en ligne, pourrait apporter une vision plus globale des tendances émotionnelles. Enfin, l'intégration de techniques d'apprentissage machine plus avancées pourrait contribuer à une classification encore plus précise des sentiments exprimés.

En outre, une idée prometteuse consisterait à entraîner le modèle en anglais, puis lorsque l'utilisateur entre un texte en français, une traduction serait effectuée avant de procéder à l'analyse des sentiments. Cette approche bilingue pourrait étendre considérablement la portée de notre application et en améliorer l'utilité pour les utilisateurs francophones.

