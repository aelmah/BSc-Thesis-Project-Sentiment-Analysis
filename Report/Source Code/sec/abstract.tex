\newpage
\chapter*{ABSTRACT}
\addcontentsline{toc}{section}{ABSTRACT}
Understanding the opinions or views of others on products or services and grasping their emotions and attitudes towards a social event has been an essential informational aspect, especially during the decision-making process.  People are highly interested in others' opinions across various domains; they seek out reviews before making a purchase or utilizing a service. \\
This report focuses on analyzing the performance of various machine learning models for sentiment classification of Twitter messages (tweets) and the natural language processing (NLP) methods employed. A comparative study of multiple machine learning algorithms is conducted, leveraging different preprocessing techniques, including lemmatization and stemming, as well as various text representation techniques applied to two datasets: one concerning sentiments (positive and negative) and the other emotions (joy, sadness, fear, etc.). This study reveals that the models' performance depends on numerous factors such as preprocessing techniques, text representation methods, the nature of datasets, and the algorithms employed. Additionally, a web application has been developed using Django to provide an intuitive user interface, demonstrating the practical utility of this approach.\\
\emph{{\textbf{Keywords:}} NLP, Sentiment Analysis, Machine Learning, Emotions, Django.}