\newpage
\chapter*{RÉSUMÉ}
\addcontentsline{toc}{section}{RÉSUMÉ}

Connaitre l'opinion ou l'avis des autres personnes sur des produits ou des services et comprendre leurs émotions et attitudes envers un événement social été un élément d'information important surtout durant le processus de décision, les gens s'intéressent énormément aux avis des autres dans les autres domaines, ils consultent les avis des autres avant d'effectuer un achat ou d'utiliser un service.\\
Ce rapport se concentre sur l'analyse des performances de divers modèles d'apprentissage automatique pour la classification des sentiments des messages Twitter (tweets) et sur les méthodes de traitement automatique du langage naturel (NLP) utilisées. Une étude comparative de plusieurs algorithmes d'apprentissage automatique est réalisée, en s'appuyant sur différentes techniques de prétraitement, notamment la lemmatisation et le stemming, ainsi que sur diverses techniques de représentation textuelle appliquées à deux ensembles de données : l'un concernant les sentiments (positif et négatif) et l'autre les émotions (joie, tristesse, peur, etc.). Cette étude révèle que la performance des modèles dépend de nombreux facteurs, tels que les techniques de prétraitement, les méthodes de représentation textuelle, la nature des ensembles de données (Complexité, volume..) et les algorithmes utilisés. En outre, une application web a été développée avec Django pour fournir une interface utilisateur intuitive, démontrant l'utilité pratique de cette approche.
\\
\emph{{\textbf{Mots clés:}} NLP, Analyse des sentiments, Apprentissage automatique, émotions, Django.}


